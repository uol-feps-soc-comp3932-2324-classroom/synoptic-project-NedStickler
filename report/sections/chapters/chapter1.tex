\justifying
\chapter{Introduction and Background Research}

% You can cite chapters by using '\ref{chapter1}', where the label must
% match that given in the 'label' command, as on the next line.
\label{chapter1}

% Sections and sub-sections can be declared using \section and \subsection.
% There is also a \subsubsection, but consider carefully if you really need
% so many layers of section structure.
\section{Introduction}

Super-resolution image reconstruction describes the task of estimating a high-resolution representation of a low-resolution image~\cite{superResOverview}. It is a fundamental problem in the field of computer vision due to the ill-posed nature of the task: there are many `correct' HR representations of a single LR image~\cite{superResChallenges,superResRemoteSensingOverview}. SR image reconstruction is utilised in many domains, including remote sensing, medical imaging, surveillance, astronomy, and more~\cite{superResRemoteSensingChallenges, superResRemoteSensingOverview, superResMedicalImaging, superResSurveillance, superResAstronomy, superResUses}.

Remote sensing is the process of gathering data on an object without making physical contact~\cite{remoteSensing}. A predominant feature of remote sensing is aerial or satellite imagery, where images of the Earth are taken from aircraft of satellites respectively~\cite{ref}. Such images have proved useful in a variety of areas, including but not limited to urban planning~\cite{remoteSensingUses}, environmental analysis, land use management, and weather prediction~\cite{remoteSensingGANsReview}. \textcolor{blue}{References needed beyond this point.}

Remote sensing provides a particularly interesting set of conditions for the super resolution problem due to the constraints introduced by imaging hardware. Atmospheric conditions, hardware resolution and [something else here] can result in undesirable image quality, making SR reconstruction techniques attractive alternatives for producing high-resolution imagery.

SR reconstruction in remote sensing has matured as a field over recent years, with a host of proposed solutions to the problem. Traditional methods can be categorised as interpolation-based or reconstruction-based, with new learning-based methods proving particularly effective in the last decade. More recently the conception of generative deep learning architectures, such as generative adversarial networks, has led to a revived interest in the problem.

This project explores the effectiveness of GAN-based solutions to the super-resolution problem in remote sensing. A modification to a previous solution is suggested with the aim of improving model performance.

% Must provide evidence of a literature review. Use sections
% and subsections as they make sense for your project.
\section{Background research}

\subsection{Remote sensing}

\subsection{Super-resolution reconstruction}

\subsection{Deep learning}

\begin{itemize}
    \item Interpolation methods
    \item Reconstruction methods
    \item Learning methods
    \item GANs
    \item SRGAN
\end{itemize}

<Section requirements. Have I\dots
\begin{itemize}
    \item Explained the problem clearly?
    \item Identified relevant areas for investigation and discussed them in the report under appropriate headings and critically?
    \item Reviewed previous attempts to solve this and similar problems?
    \item Justified any claims made using credible primary or secondary sources?
\end{itemize}
>