\chapter{Introduction and Background Research}

% You can cite chapters by using '\ref{chapter1}', where the label must
% match that given in the 'label' command, as on the next line.
\label{chapter1}

% Sections and sub-sections can be declared using \section and \subsection.
% There is also a \subsubsection, but consider carefully if you really need
% so many layers of section structure.
\section{Introduction}

Super-resolution image reconstruction describes the task of estimating a high-resolution representation of a low-resolution image~\cite{superResOverview}. It is a fundamental problem in the field of computer vision due to the ill-posed nature of the task: there are many `correct' HR representations of a single LR image~\cite{superResChallenges,superResRemoteSensingOverview}. SR image reconstruction is utilised in many domains, including remote sensing, medical imaging, surveillance, astronomy, and more~\cite{superResRemoteSensingChallenges, superResRemoteSensingOverview, superResMedicalImaging, superResSurveillance, superResAstronomy, superResUses}.

Remote sensing is the process of gathering data on an object without making physical contact~\cite{remoteSensing}. A predominant feature of remote sensing is aerial or satellite imagery, where images of the Earth are taken from aircraft of satellites respectively~\cite{ref}. Such images have proved useful in a variety of areas, including but not limited to urban planning~\cite{remoteSensingUses}, environmental analysis, land use management, and weather prediction~\cite{remoteSensingGANsReview}. \textcolor{blue}{References needed beyond this point.}

Remote sensing provides a particularly interesting set of conditions for the super resolution problem due to the constraints introduced by imaging hardware. Atmospheric conditions, hardware resolution and [something else here] can result in undesirable image quality, making SR reconstruction techniques attractive alternatives for producing high-resolution imagery.

SR reconstruction in remote sensing has matured as a field over recent years, with a host of proposed solutions to the problem. Traditional methods can be categorised as interpolation-based or reconstruction-based, with new learning-based methods proving particularly effective in the last decade. More recently the conception of generative deep learning architectures, such as generative adversarial networks, has led to a revived interest in the problem.

This project explores the effectiveness of GAN-based solutions to the super-resolution problem in remote sensing. A modification to a previous solution is suggested with the aim of improving model performance.

The problem: Super resolution in remote sensing. There exists a variety of solutions. Literature review on those solutions. Take SRGAN and improve it by improving the perceptual loss metric.

% Must provide evidence of a literature review. Use sections
% and subsections as they make sense for your project.
\section{Background research}

\subsection{Remote sensing}

\subsection{Super-resolution reconstruction}

\subsection{Neural networks}

\subsection{Convolutional neural networks}

\subsection{Deep learning}

\subsection{Generative deep-learning}

\subsection{Generative adversarial networks}

\subsection{SRGAN}
The SRGAN model, proposed by Ledig et al.\ in 2017, introduced a new approach to the super-resolution problem~\cite{srgan}. By employing a generative adversarial network architecture researchers were able to achieve state-of-the-art results, consequently kickstarting the large-scale development of GAN-based super-resolution solutions. Since then, numerous adaptations of the SRGAN model have been proposed with many surpassing the SR reconstruction capabilities of SRGAN~\cite{models}. Regardless, SRGAN remains the most important and influential GAN-based SR reconstruction model.

The SRGAN model has two components, the generator and the discriminator. The generator model is responsible for upscaling LR imagery to produce the SR output, and the discriminator is responsible for providing a goal for the generator during adversarial training. The generator of the SRGAN, named SRResNet, was also proposed by Ledig et al. SRResNet employs a residual block structure to increase the resolution of the input imagery, where a series of residual blocks with skip connections learn the features of the dataset. The residual blocks are followed by two upsample operations executed with the pixel shuffle technique. During training, the model learns the best features to extract from the training dataset in order to then upsample. New imagery can then be passed through SRResNet where the learned filters are applied, the upsampling operation is executed, and SR imagery is produced. SRResNet generates sufficient SR imagery independent of the discriminator component of SRGAN, however the results often fail to capture high-frequency information such as texture and detailed objects~\cite{srgan}. The discriminator is introduced to further improve the SRResNet model, where adversarial training and the perceptual loss metric aid in generating imagery where the high-frequency components are not lost, and the final output is as close as possible to the original HR image. The discriminator component follows a stereotypical CNN classifier architecture, where the image features are captured and reduced, and finally flattened into a fully connected sequence that reduces to a single neuron. The output of the neuron is passed through a sigmoid function and the output represents the classification of real or fake for the generated images.

\subsection{Additional SR GAN models}

<Section requirements. Have I\dots
\begin{itemize}
    \item Explained the problem clearly?
    \item Identified relevant areas for investigation and discussed them in the report under appropriate headings and critically?
    \item Reviewed previous attempts to solve this and similar problems?
    \item Justified any claims made using credible primary or secondary sources?
\end{itemize}
>