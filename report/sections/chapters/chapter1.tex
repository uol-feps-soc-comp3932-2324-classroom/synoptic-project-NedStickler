\chapter{Introduction and Background Research}

% You can cite chapters by using '\ref{chapter1}', where the label must
% match that given in the 'label' command, as on the next line.
\label{chapter1}

<Section requirements. Have I\dots
\begin{itemize}
    \item Explained the problem clearly?
    \item Identified relevant areas for investigation and discussed them in the report under appropriate headings and critically?
    \item Reviewed previous attempts to solve this and similar problems?
    \item Justified any claims made using credible primary or secondary sources?
\end{itemize}
>

% Sections and sub-sections can be declared using \section and \subsection.
% There is also a \subsubsection, but consider carefully if you really need
% so many layers of section structure.
\section{Introduction}

\begin{itemize}
    \item Explain SISR
    \item Explain remote sensing
    \item Explain why SR is useful in remote sensing
\end{itemize}

Single-image super-resolution (SISR) is a fundamental problem in the field of computer vision. It refers to the process where a single low-resolution (LR) image is reconstructed to create a high-resolution (HR) representation, with the aim of better defining image features [ref]. Super-resolution (SR) reconstruction is an ill-posed problem due to the large solution space for any one image: there are many SR representations that are considered correct for a single LR image [ref].

% Must provide evidence of a literature review. Use sections
% and subsections as they make sense for your project.
\section{Literature review}

\begin{itemize}
    \item Interpolation methods
    \item Reconstruction methods
    \item Learning methods
    \item GANs
    \item SRGAN
\end{itemize}

Generative adversarial networks (GANs) are family of neural networks with specific purpose of generating imagery.